\documentclass[10pt,aspectratio=43,serif]{beamer}		
%\documentclass[handout,t]{beamer}

\batchmode
% \usepackage{pgfpages}
% \pgfpagesuselayout{4 on 1}[letterpaper,landscape,border shrink=5mm]

\usepackage{amsmath,amssymb,enumerate,epsfig,bbm,calc,color,ifthen,capt-of,multimedia,hyperref,ctex,listings}

\usetheme{Berlin}

\DefineNamedColor{named}{socodingblue}    {rgb}{0.296875,0.421875,0.69921875}

\mode<presentation>
\setbeamercolor*{palette primary}{bg=socodingblue,fg=white}
\setbeamercolor*{palette secondary}{bg=socodingblue,fg=white}
\setbeamercolor*{palette tertiary}{bg=socodingblue,fg=white}
\setbeamercolor*{palette quaternary}{bg=socodingblue,fg=white}
\setbeamercolor*{structure}{fg=socodingblue,bg=white}
\setbeamercolor{frametitle}{bg=socodingblue,fg=white}
\mode<all>

\setbeamertemplate{headline}
{%
	\begin{beamercolorbox}[colsep=1.5pt]{upper separation line head}
	\end{beamercolorbox}
	\begin{beamercolorbox}{section in head/foot}
		\vskip2pt\insertnavigation{\paperwidth}\vskip2pt
	\end{beamercolorbox}%
	\begin{beamercolorbox}[colsep=1.5pt]{lower separation line head}
	\end{beamercolorbox}
}

\setbeamertemplate{footline}[frame number]{}

%\setbeamertemplate{navigation symbols}{}

\setbeamertemplate{footline}{}

\catcode`\。=\active
\newcommand{。}{.}

\title{数论入门}

\author{Jialin Rong}
\institute{South China Normal University}
\date{\today}

\AtBeginSection[]
{
	\begin{frame}<beamer>
		\frametitle{目录}
		\tableofcontents[currentsection]
	\end{frame}
}
\beamerdefaultoverlayspecification{<+->}

\begin{document}

% -----------------------------------------------------------------------------
	
\frame{\titlepage}
	
%\section[目录]{}
\begin{frame}{目录}	
	\tableofcontents
\end{frame}

\begin{frame}{声明}	
	本演示文档可能包含大量不严谨的表述。
\end{frame}
	
% -----------------------------------------------------------------------------

\section{基础知识}
\begin{frame}{整除}
	
	存在整数 $q$ 使得 $b=aq$。
		
	\begin{itemize}
		\item $a \mid b$:$a$ 整除 $b$
		\item $a \nmid b$:$a$ 不整除 $b$
	\end{itemize}

\end{frame}

% -----------------------------------------------------------------------------

\begin{frame}{带余除法}
	
	给定 $a,b$,一定存在唯一一对 $q,r$(都是整数)
	
	\begin{itemize}
		\item $a=bq+r$
		\item $0 \leq r < |b|$
	\end{itemize}
	
\end{frame}

% -----------------------------------------------------------------------------

\begin{frame}{整除的性质}
	
	都是整数...
	
	\begin{itemize}
		\item $a\mid b$ 且 $b\mid c \Longrightarrow a\mid c$
		\item $a\mid b$ 且 $a \mid c  \Longrightarrow$ 对任意 $x,y$ 有 $a\mid bx + cy$
		\item $m \neq 0$,$a\mid b \Longleftrightarrow ma \mid mb$
		\item $a\mid b$ 且 $b \mid a \Longrightarrow a=\pm b$ 
		\item $b\neq 0$ 且 $a \mid b \Longrightarrow |a| \leq |b|$ 
	\end{itemize}	
		
\end{frame}
	

% -----------------------------------------------------------------------------

\begin{frame}{整除的性质}
	
	\begin{itemize}
		\item $a \equiv a \pmod m$
		\item $a \equiv b \pmod m\Longleftrightarrow b \equiv a \pmod m$
		\item $a \equiv b \pmod m,b \equiv c \pmod m \Longrightarrow a \equiv c \pmod m$
		\item $a \equiv b \pmod m,c \equiv d \pmod m \Longrightarrow a\pm c \equiv b\pm d \pmod m$
		\item $a \equiv b \pmod m,c \equiv d \pmod m \Longrightarrow ac \equiv bd \pmod m$
	\end{itemize}
	
\end{frame}

% -----------------------------------------------------------------------------

\begin{frame}{整除的性质}
	
	很熟悉的...
	
	\begin{itemize}
		\item $(a+b) \bmod p = (a\bmod p+b\bmod p)\bmod p$
		\item $(a-b) \bmod p = (a\bmod p-b\bmod p + p)\bmod p$
		\item $(a \times b) \bmod p = (a \bmod p )( b \bmod p)\bmod p$
		\item 除法?
		\item $\frac{100}{50} \bmod 20 \neq \frac{100 \bmod 20}{50 \bmod 20} \bmod 20$
	\end{itemize}
	
\end{frame}

% -----------------------------------------------------------------------------

\section{快速幂、质数判断、筛法、唯一分解定理}

% -----------------------------------------------------------------------------

\section{GCD、ExGCD}
\begin{frame}{GCD 性质}
	
	\begin{itemize}
		\item $(a,b)=(a,b+xa)$
		\item $m>0,m(a_1,a_2,\cdots,a_k)=(ma_1,ma_2,\cdots,ma_k)$
		\item $(\frac{a}{(a,b)},\frac{b}{(a,b)})=1$
		\item $(a_1,a_2,a_3,\cdots,a_k)=((a_1,a_2),a_3,\cdots,a_k)$
		\item $(m,a)=1$ 则 $(m,ab)=(m,b)$
		\item $(a^k,b^k)=(a,b)^k$
		\item $(m,a)=1$,若 $m\mid ab$,则 $m\mid b$
	\end{itemize}
	
\end{frame}

% -----------------------------------------------------------------------------

\begin{frame}{GCD 求法}
	
	\begin{itemize}
		\item $(a,b)=(a,b+xa)$
		\item 令 $a=bq+r$
		\item $(a,b)=(bq+r,b)=(r,b)=(a \bmod b,b)$
	\end{itemize}
	
\end{frame}

% -----------------------------------------------------------------------------

\begin{frame}{ExGCD}
	
	给定 $a,b$,求解一组 $x,y$ 满足 $ax+by = \gcd(a, b)$
	
	\begin{itemize}
		\item 设 $ax_1+by_1=\gcd(a,b)$
		\item 设 $bx_2+(a\bmod b)y_2=\gcd(b,a\bmod b)$
		\item 因为 $\gcd(a,b)=\gcd(b,a\bmod b)$
		\item 所以 $ax_1+by_1=bx_2+(a\bmod b)y_2$
		\item 因为 $a \bmod b=a-\lfloor \frac{a}{b}\rfloor \times b$
		\item 所以 $ax_1+by_1=bx_2+(a-(\lfloor\frac{a}{b}\rfloor\times b))y_2$
		\item 整理得 $ax_1+by_1=ay_2+bx_2-\lfloor\frac{a}{b}\rfloor\times by_2=ay_2+b(x_2-\lfloor\frac{a}{b}\rfloor y_2)$
		\item 所以 $x_1=y_2,y_1=x_2-\lfloor\frac{a}{b}\rfloor y_2$
	\end{itemize}
	
	
\end{frame}

% -----------------------------------------------------------------------------

\section{逆元}
\begin{frame}{定义}
	
	对于同余方程 $ax\equiv 1 \pmod m$,如果存在解 $x$,\\
	
	则称 $x$ 是 $a$ 在模 $m$ 意义下的逆元。
	
	~\\
	
	$a$ 模 $m$ 有逆元 $\Longleftrightarrow$ $a$ 和 $m$ 互质(为什么)
	
	~\\
	
	\begin{itemize}
		\item ExGCD(模不一定是质数)
		\item 费马小定理(模是质数)
	\end{itemize}
	
\end{frame}

% -----------------------------------------------------------------------------

\begin{frame}{ExGCD 求法}
	
	\begin{itemize}
		\item $a \cdot inv(a) \equiv 1 \pmod m$
		\item $a \cdot inv(a) + m \cdot k  = 1$
		\item $a \cdot inv(a) + m \cdot k  = \gcd(a,m)$
	\end{itemize}
	
\end{frame}

% -----------------------------------------------------------------------------

\begin{frame}{技巧}
	
	\begin{itemize}
		\item 求阶乘的逆元:$inv(i!)=inv((i+1)!) \times (i+1) \bmod p$
		\item 线性求逆元:$inv(i)=(p-\lfloor\frac{p}{i}\rfloor) \times inv(p \bmod i) \bmod p$
	\end{itemize}

	~\\
	
	\begin{itemize}
		\item $\lfloor\frac{p}{i}\rfloor \times i+ (p \bmod i) \equiv 0 \pmod p$
		\item $-\lfloor\frac{p}{i}\rfloor \times i\equiv p \bmod i \pmod p$
		\item $-\lfloor\frac{p}{i}\rfloor \times inv(p \bmod i) \equiv inv(i) \pmod p$
	\end{itemize}
	
\end{frame}

% -----------------------------------------------------------------------------

\section{欧拉函数、欧拉定理、费马小定理}
\begin{frame}{同余类}
	
	通过一个整数模 $n$ 的余数,我们可以把所有整数分成 $n$ 类
	
	$$
	\overline{r}_n=\{m\in \mathbb{Z} \mid mn+r\}
	$$
	
	为模 $n$ 余 $r$ 的同余类
	
\end{frame}

% -----------------------------------------------------------------------------

\begin{frame}{完全剩余系}
	
	从 $\overline{0}_n, \overline{1}_n, \overline{2}_n, \dots,\overline{(n-1)}_n$ 中各挑出一个数就组成了一个模 $n$ 的完全剩余系 $R_n$
	
	$$
	R_n = \{r_0, r_1, \dots, r_{n-1}\}
	$$
	
	其中 $r_0 \in \overline0_n, r_1 \in \overline1_n, r_2 \in \overline2_n, \dots, r_{n-1} \in \overline{(n-1)}_n$
	
	~\\
	
	$R_n = \{0, 1, \dots, n-1\}$ 为模 $n$ 的最小非负完全剩余系
	
\end{frame}

% -----------------------------------------------------------------------------

\begin{frame}{缩剩余系}
	
	取一个模 $n$ 的完全剩余系 $R_n$,取出里面所有和 $n$ 互质的数,这些数组成一个模 $n$ 的缩剩余系(缩系),记为 $\Phi_n$
	
	$$
	\Phi_n = \{c_1, c_2, \dots, c_{\varphi(n)} \}
	$$
	
	~\\
	
	每一个模 $n$ 的缩剩余系有相同数量的元素(为什么)
	
\end{frame}

% -----------------------------------------------------------------------------

\begin{frame}{欧拉函数}
	
	$\varphi(x)=$小于等于 $x$ 和 $x$ 互质的数的个数
	
	\begin{itemize}
		\item $\varphi(1)=1$
		\item $p$ 是质数,$\varphi(p)=p-1$
		\item $p$ 是质数,$\varphi(p^k)=p^k-p^{k-1}$
		\item 积性函数:$\gcd(m,n)=1,\varphi(mn)=\varphi(m)\times\varphi(n)$
		\item $\varphi(n) = n \times \prod_{i = 1}^s{\dfrac{p_i - 1}{p_i}}$
	\end{itemize}
	
\end{frame}

% -----------------------------------------------------------------------------

\begin{frame}{欧拉定理}
	
	如果正整数 $n$ 和整数 $a$ 互质,那么就有
	
	$$a^{\varphi(n)}\equiv 1 \pmod{n}$$
	
	\begin{itemize}
		\item 反证法证明 $a\Phi_n$ 是一个模 $n$ 的缩系
		\item $\prod_{i=1}^{\varphi(n)} c_i\equiv \prod_{i=1}^{\varphi(n)} ac_i = a^{\varphi(n)}\prod_{i=1}^{\varphi(n)} c_i \pmod{n}$
		\item $\gcd\left(n, \prod_{i=1}^{\varphi(n)} c_i \right)=1$
	\end{itemize}
	
\end{frame}

% -----------------------------------------------------------------------------

\begin{frame}{费马小定理}
	   
	若 $p$ 为素数,$\gcd(a,p)=1$,则 $a^{p - 1} \equiv 1 \pmod{p}$。
	
	\begin{itemize}
		\item $a \times a^{p-2} \equiv 1 \pmod{p}$
	\end{itemize}
	
\end{frame}

% -----------------------------------------------------------------------------

\end{document}
