\documentclass[10pt,aspectratio=43,serif]{beamer}		
%\documentclass[handout,t]{beamer}

\batchmode
% \usepackage{pgfpages}
% \pgfpagesuselayout{4 on 1}[letterpaper,landscape,border shrink=5mm]

\usepackage{amsmath,amssymb,enumerate,epsfig,bbm,calc,color,ifthen,capt-of,multimedia,hyperref,ctex,listings}

\usetheme{Berlin}

\setbeamertemplate{headline}
{%
  \begin{beamercolorbox}[colsep=1.5pt]{upper separation line head}
  \end{beamercolorbox}
  \begin{beamercolorbox}{section in head/foot}
    \vskip2pt\insertnavigation{\paperwidth}\vskip2pt
  \end{beamercolorbox}%
  \begin{beamercolorbox}[colsep=1.5pt]{lower separation line head}
  \end{beamercolorbox}
}

\setbeamertemplate{footline}[frame number]{}

%\setbeamertemplate{navigation symbols}{}

\setbeamertemplate{footline}{}

\catcode`\。=\active
\newcommand{。}{.}

\title{题目分享(一)}
\subtitle{顺便线段树、单调栈扫盲}
\author{Jialin Rong}
\institute{South China Normal University}
\date{\today}

\AtBeginSection[]
{
  \begin{frame}<beamer>
    \frametitle{目录}
    \tableofcontents[currentsection]
  \end{frame}
}
\beamerdefaultoverlayspecification{<+->}

\begin{document}
% -----------------------------------------------------------------------------

\frame{\titlepage}

%\section[目录]{}
\begin{frame}{目录}
  \tableofcontents
\end{frame}

% -----------------------------------------------------------------------------

\begin{frame}{解决什么问题}
\begin{itemize}
  \item 线段树:给定区间求最值.
  \item 单调栈:给定最值求区间.
\end{itemize}
\end{frame}

% -----------------------------------------------------------------------------

\section{线段树}
\begin{frame}{是老问题了}

\begin{block}{维护一个数组 $(0 \leq a_i \leq 10^9)$,实现下面的操作:}
\begin{itemize}
  \item 将 $a_i$ 设置为 $v$。
  \item 求 $\sum_{i=L}^{R}\sum_{j=L}^{i}a_{i}a_{j}$ 的值,对 $10^9+7$ 取模输出。
\end{itemize}

\end{block}

\begin{itemize}
  \item 如果数组大小和操作次数 $1 \leq n,q \leq 100$?
  \item (2020 SCNUCS-N)如果数组大小和操作次数 $1 \leq n,q \leq 5 \times 10^5$?
  \item 如果不考虑修改?我们就来实现一下查询?
\end{itemize}

\end{frame}

% -----------------------------------------------------------------------------

\begin{frame}{稍微变一下}

\begin{block}{维护一个数组 $(0 \leq a_i \leq 10^9)$,实现下面的操作:}
\begin{itemize}
  \item 将 $a_i$ 设置为 $v$。
  \item 求 $\max_{i=L}^{R}a_i$ 的值。
\end{itemize}

\end{block}

\begin{itemize}
  \item 如果数组大小和操作次数 $1 \leq n,q \leq 100$?
  \item 如果数组大小和操作次数 $1 \leq n,q \leq 10^5$?
\end{itemize}

\end{frame}

% -----------------------------------------------------------------------------

\begin{frame}{如何预处理?}

\begin{figure}
  \begin{center}
    \includegraphics[scale=0.19]{figures/seg.png}
  \end{center}
\end{figure}

\end{frame}

% -----------------------------------------------------------------------------


\begin{frame}[fragile]
\frametitle{如何预处理?}

\begin{figure}
  \begin{center}
    \includegraphics[scale=0.21]{figures/build.png}
  \end{center}
\end{figure}

\begin{figure}
  \begin{center}
    \includegraphics[scale=0.19]{figures/seg.png}
  \end{center}
\end{figure}

\end{frame}

% -----------------------------------------------------------------------------

\begin{frame}[fragile]
\frametitle{如何预处理?}

\begin{figure}
  \begin{center}
    \includegraphics[scale=0.21]{figures/buildcode.png}
  \end{center}
\end{figure}

\end{frame}

% -----------------------------------------------------------------------------

\begin{frame}{如何查询?}

\begin{figure}
  \begin{center}
    \includegraphics[scale=0.19]{figures/seg.png}
  \end{center}
\end{figure}

\end{frame}

% -----------------------------------------------------------------------------

\begin{frame}{如何查询?}

\begin{figure}
  \begin{center}
    \includegraphics[scale=0.20]{figures/query.png}
  \end{center}
\end{figure}

\begin{figure}
  \begin{center}
    \includegraphics[scale=0.19]{figures/seg.png}
  \end{center}
\end{figure}

\end{frame}

% -----------------------------------------------------------------------------

\begin{frame}{如何查询?}

\begin{figure}
  \begin{center}
    \includegraphics[scale=0.25]{figures/querycode.png}
  \end{center}
\end{figure}

\begin{description}
  \item \begin{block}{把这个收录到我的模板里,比赛时直接照搬,会有什么隐患吗?}
\begin{itemize}
  \item 考虑 $a_i$ 都为负数时 $res$ 的初始值。
\end{itemize}
\end{block}
\end{description}

\end{frame}

% -----------------------------------------------------------------------------

\begin{frame}{单点更新?}

\begin{figure}
  \begin{center}
    \includegraphics[scale=0.19]{figures/seg.png}
  \end{center}
\end{figure}

\end{frame}

% -----------------------------------------------------------------------------

\begin{frame}{单点更新?}

\begin{figure}
  \begin{center}
    \includegraphics[scale=0.20]{figures/update.png}
  \end{center}
\end{figure}

\begin{figure}
  \begin{center}
    \includegraphics[scale=0.19]{figures/seg.png}
  \end{center}
\end{figure}

\end{frame}

% -----------------------------------------------------------------------------

\begin{frame}{单点更新?}

\begin{figure}
  \begin{center}
    \includegraphics[scale=0.20]{figures/updatecode.png}
  \end{center}
\end{figure}

\end{frame}

% -----------------------------------------------------------------------------



\section{单调栈}
\begin{frame}{AK 杯前应该做过吧...}



\begin{block}{小鱼比可爱}
人比人,气死人;鱼比鱼,难死鱼。小鱼最近参加了一个“比可爱”比赛,比的是每只鱼的可爱程度。参赛的鱼被从左到右排成一排,头都朝向左边,然后每只鱼会得到一个整数数值,表示这只鱼的可爱程度,很显然整数越大,表示这只鱼越可爱,而且任意两只鱼的可爱程度可能一样。由于所有的鱼头都朝向左边,所以每只鱼只能看见在它左边的鱼的可爱程度,它们心里都在计算,在自己的眼力范围内有多少只鱼不如自己可爱呢。请你帮这些可爱但是鱼脑不够用的小鱼们计算一下。
\end{block}

\begin{itemize}
  \item (洛谷 P1428)如果鱼的数量 $1 \leq n \leq 100$。
  \item 如果鱼的数量 $1 \leq n \leq 10^5$。
  \begin{itemize}
    \item 逆序对 - 归并排序 / 树状数组
  \end{itemize}
  \item 那如果现在我比身高而不是比可爱呢?$ 1 \leq n \leq 100$。
  \item 还是比身高,$1 \leq n \leq 10^5$。
\end{itemize}

\end{frame}

% -----------------------------------------------------------------------------

\begin{frame}{小鱼比身高}

$$[1,1,4,5,1,4,1,1,4,5,1,4]$$

\begin{itemize}
  \item 在 $h_3$ 出现后,$h_1,h_2$ 就没有用处了。
  \item 在 $h_4$ 出现后,$h_3$ 也没有用处了。
  \item 在 $h_6$ 出现后,$h_5$ 也没有用处了。
  \item 在 $h_9$ 出现后,$h_7,h_8$ 也没有用处了。
  \item 在 $h_{10}$ 出现后,$h_6,h_9$ 也没有用处了。
  \item 在 $h_{12}$ 出现后,$h_{11}$ 也没有用处了。
\end{itemize}

\end{frame}

% -----------------------------------------------------------------------------

\begin{frame}{利用价值}

$$[1,1,4,5,1,4,1,1,4,5,1,4]$$

\begin{itemize}
  \item 为什么失去了利用价值?
  \item 不可能影响 $c_i$ 的计算。
  \item 不再可能成为第一个 $ \geq h_i$ 的数。
\end{itemize}

\end{frame}

% -----------------------------------------------------------------------------

\begin{frame}{实现}

\begin{itemize}
  \item 怎么知道哪些数还有价值?
\end{itemize}

\begin{block}{实现}
\begin{itemize}
  \item 比新出现的数小的数统统弹栈。
  \item 新出现的数入栈,同时得到答案。
\end{itemize}
\end{block}


\begin{itemize}
  \item 为什么会满足单调性?
  \item 怎么得到答案?
  \item 为什么不是小于等于?
  \item 时间复杂度?
\end{itemize}

\end{frame}

% -----------------------------------------------------------------------------

\begin{frame}{实现}

\begin{figure}
  \begin{center}
    \includegraphics[scale=0.22]{figures/stack.png}
  \end{center}
\end{figure}


\end{frame}

% -----------------------------------------------------------------------------

\section{简单应用}
\begin{frame}{题目}

(2021 牛客寒假多校 2A)对于一个数组,定义 $RMQ(L,R)$ 是数组下标区间 $[L,R]$ 的最大值。\\

~\\


现在给定长度为 $n \ (1 \leq n \leq 10^5)$ 的全排列数组 $a[ ] \ (1 \leq a_i \leq n)$,接下来有 $m \ (1\leq m \leq 2 \times 10^5)$ 次询问。\\

~\\

每次询问中,给定 $k \ (1 \leq k \leq 10^5)$ 个不同的数 $p[ ] \ (1 \leq p_i \leq n)$,你可以从中有放回地随机抽取两个数 $p_x, p_y$,不妨设 $p_x \leq p_y$。\\

~\\

现在问你对于 $a[]$,$RMQ(p_x,p_y)$ 有哪几种情况,这些情况出现的概率各是多大。\\

~\\

数据保证 $\sum k \leq 2 \times 10^5$。

\end{frame}

% -----------------------------------------------------------------------------

\begin{frame}{思路}

现在假设 $p[]$ 已经从小到大排序过了,可能成为 $RMQ(p_x,p_y)$ 的数就只有下面两类数:

\begin{itemize}
  \item 如果两次抽取的下标相同:$a_{p_{1}}$, $a_{p_{2}}$, $a_{p_{3}}$, ..., $a_{p_{i}}$, ..., $a_{p_{k}}$。
  \item 如果两次抽取的下标不相同:$RMQ(p_1,p_2)$, $RMQ(p_2,p_3)$, $RMQ(p_3,p_4)$, ..., $RMQ(p_i,p_{i+1})$, ..., $RMQ(p_{k-1},p_{k})$。
\end{itemize}

\end{frame}

% -----------------------------------------------------------------------------


\begin{frame}{思路}

\begin{figure}
    \begin{center}
      \includegraphics[scale=0.17]{figures/prob.png}
    \end{center}
  \end{figure}

\begin{itemize}
  \item 对于数 $a_i$,$L$ 和 $R$ 取到什么位置会有 $RMQ(L,R)=a_i$?
  \item 如何计算对答案的贡献?
  \item 考虑 $RMQ(p_{i-1},p_{i})=RMQ(p_i,p_{i+1})$ 的情况?
\end{itemize}

\end{frame}

% -----------------------------------------------------------------------------

\end{document}
