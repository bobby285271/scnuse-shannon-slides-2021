\documentclass[10pt,aspectratio=43,serif]{beamer}		
%\documentclass[handout,t]{beamer}

\batchmode
% \usepackage{pgfpages}
% \pgfpagesuselayout{4 on 1}[letterpaper,landscape,border shrink=5mm]

\usepackage{amsmath,amssymb,enumerate,epsfig,bbm,calc,color,ifthen,capt-of,multimedia,hyperref,ctex,listings}

\usetheme{Berlin}

\DefineNamedColor{named}{socodingblue}    {rgb}{0.296875,0.421875,0.69921875}

\mode<presentation>
\setbeamercolor*{palette primary}{bg=socodingblue,fg=white}
\setbeamercolor*{palette secondary}{bg=socodingblue,fg=white}
\setbeamercolor*{palette tertiary}{bg=socodingblue,fg=white}
\setbeamercolor*{palette quaternary}{bg=socodingblue,fg=white}
\setbeamercolor*{structure}{fg=socodingblue,bg=white}
\setbeamercolor{frametitle}{bg=socodingblue,fg=white}
\mode<all>

\setbeamertemplate{headline}
{%
  \begin{beamercolorbox}[colsep=1.5pt]{upper separation line head}
  \end{beamercolorbox}
  \begin{beamercolorbox}{section in head/foot}
    \vskip2pt\insertnavigation{\paperwidth}\vskip2pt
  \end{beamercolorbox}%
  \begin{beamercolorbox}[colsep=1.5pt]{lower separation line head}
  \end{beamercolorbox}
}

\setbeamertemplate{footline}[frame number]{}

%\setbeamertemplate{navigation symbols}{}

\setbeamertemplate{footline}{}

\catcode`\。=\active
\newcommand{。}{.}

\title{平面计算几何基础}
\author{Jialin Rong}
\institute{South China Normal University}
\date{\today}

\AtBeginSection[]
{
  \begin{frame}<beamer>
    \frametitle{目录}
    \tableofcontents[currentsection]
  \end{frame}
}
\beamerdefaultoverlayspecification{<+->}

\begin{document}
% -----------------------------------------------------------------------------

\frame{\titlepage}

%\section[目录]{}
\begin{frame}{目录}
  \tableofcontents
\end{frame}

% -----------------------------------------------------------------------------

\section{点线关系}
\begin{frame}{从熟悉的开始}
	
	$$ax+by+c=0$$
	$$(x_0,y_0)$$
	
	\begin{itemize}
		\item $b \neq 0$
		\item $y=-\frac{a}{b}x-\frac{c}{b}$
		\item $y_0 > -\frac{a}{b}x_0-\frac{c}{b}$
		\item $b>0, ax_0+by_0+c>0$
		\item $b<0$
	\end{itemize}
	
\end{frame}

% -----------------------------------------------------------------------------

\begin{frame}{点积}
	\begin{description}
		\item[只考虑二维平面下的情形...]
	\end{description}
	
	\begin{itemize}
		\item 求出来是一个数量
		\item $\boldsymbol{a}=(x_1,y_1)$
		\item $\boldsymbol{b}=(x_2,y_2)$
		\item $\boldsymbol{a}\cdot \boldsymbol{b}=x_1x_2+y_1y_2$
		\item $\boldsymbol{a}\cdot \boldsymbol{b}=|\boldsymbol{a}||\boldsymbol{b}|\cos\theta$
		\item 投影相乘
	\end{itemize}

	\begin{block}{怎么推的?}
		\begin{itemize}
			\item $\boldsymbol{c}=\boldsymbol{a}-\boldsymbol{b}$
			\item $\boldsymbol{c}^2=\boldsymbol{a}^2+\boldsymbol{b}^2-2|\boldsymbol{a}||\boldsymbol{b}|\cos\theta$
			\item $(\boldsymbol{a}-\boldsymbol{b})(\boldsymbol{a}-\boldsymbol{b})$
		\end{itemize}
	\end{block}
	
\end{frame}

% -----------------------------------------------------------------------------

\begin{frame}{叉积}
	\begin{description}
		\item[只考虑二维平面下的情形...]
	\end{description}
	
	\begin{itemize}
		\item 求出来是一个向量,方向在 $z$ 轴
		\item $\boldsymbol{a}=(x_1,y_1)$
		\item $\boldsymbol{b}=(x_2,y_2)$
		\item $|\boldsymbol{a}\times \boldsymbol{b}|=|x_1y_2-x_2y_1|$
		\item $|\boldsymbol{a}\times \boldsymbol{b}|=|\boldsymbol{a}||\boldsymbol{b}|\sin\theta$
		\item 平行四边形面积
		\item 右手定则:从 $\boldsymbol{a}$ 以不超过 $180$ 度的转角转向 $\boldsymbol{b}$
		\item 用数值的正负判断叉乘后向量的方向
		\item 用数值的正负判断向量位置关系
	\end{itemize}

	\begin{block}{怎么推的?}
		\begin{itemize}
			\item 平方一下?
		\end{itemize}
		
	\end{block}
	
\end{frame}

% -----------------------------------------------------------------------------

\begin{frame}{方向判断}
	
	\begin{block}{举例}
		\begin{itemize}
			\item $A(0,0), B(4,0), M(1,2), N(3,4), O(3,-4)$
		\end{itemize}
	\end{block}
	
	\begin{itemize}
		\item $\vec{AB}\times \vec{AM}$
		\item $8$
		\item $\vec{AB}\times \vec{AN}$
		\item $16$
		\item $\vec{AB}\times \vec{AO}$
		\item $-16$
	\end{itemize}
	
\end{frame}

% -----------------------------------------------------------------------------

\begin{frame}{代码实现}
	
	\begin{block}{来点 C++}
		结构体、构造函数、重载运算符
	\end{block}
	
	\begin{itemize}
		\item $P(x,y)$
		\item 加减、常数乘
		\item 判等 $eps$
		\item 模长
		\item 点积、叉积
	\end{itemize}
	
\end{frame}

% -----------------------------------------------------------------------------

\begin{frame}{应用}
	
	\begin{description}
		\item[判断一个点 $P$ 在不在三角形 $ABC$(四个点的坐标已知)?]
	\end{description}
	
	\begin{itemize}
		\item 面积 $S_{ABP}+S_{ACP}+S_{BCP}=S_{ABC}$ 
		\item $P,A$在$BC$的同侧、$P,B$在$AC$的同侧、$P,C$在$AB$的同侧
		\item $\vec{PA}\times \vec{PB}$、$\vec{PB}\times \vec{PC}$、$\vec{PC}\times \vec{PA}$ 同号
	\end{itemize}
	
\end{frame}

% -----------------------------------------------------------------------------

\section{面积}
\begin{frame}{从熟悉的开始}
	
	\begin{description}
		\item[给定顶点坐标...]
	\end{description}
	
	\begin{itemize}
		\item 三角形面积?
		\item 平行四边形面积?
		\item 四边形面积?
		\item 任意凸多边形?
	\end{itemize}
	
\end{frame}

% -----------------------------------------------------------------------------

\begin{frame}{任意多边形面积}
	
	\begin{description}
		\item[$n$ 边形,时钟序标号 $A_i(x_i,y_i)$]
	\end{description}
	
	\begin{itemize}
		\item 还是若干三角形
		\item $S=\frac{1}{2} |\sum_{i=1}^{n}\left(x_{i} y_{i+1}-x_{i+1} y_{i}\right)|$
		\item $x_{n+1}=x_1,y_{n+1}=y_1$
		\item $A_1(3,4),A_2(5,11),A_3(12,8),A_4(9,5),A_5(5,6)$
		\item $S= 30$
		\item 数学归纳法
	\end{itemize}
	
\end{frame}

% -----------------------------------------------------------------------------

\end{document}
